\documentclass[14pt]{beamer}

%-------------------------------------------------
%   THEMES & PACKAGES
%-------------------------------------------------
\usetheme[progressbar=frametitle]{metropolis}
\usepackage{graphicx}

%-------------------------------------------------
%   TITLE
%-------------------------------------------------
\title{Software Development Project}
\subtitle{Progress Report}
\date{June 20, 2016}
\author{Jose Mayoral, Octavio Arriaga, Minh Nguyen}
\titlegraphic{\hfill\includegraphics[height=0.7cm]{images/h-brs-logo.jpg}}

%-------------------------------------------------
%   BEGIN
%-------------------------------------------------
\begin{document}

%-------------------------------------------------
\maketitle

%-------------------------------------------------
%   Features
%-------------------------------------------------
\begin{frame}{The multi\_map\_navigation Package}
    \begin{alertblock}{Features}
        \begin{itemize}
            \item Switching between multiple maps
            \item Wormhole feature for traversal between maps
        \end{itemize}
    \end{alertblock}
\end{frame}

%-------------------------------------------------
%   Integration
%-------------------------------------------------
\begin{frame}{Integration}
    \begin{alertblock}{Integration with ROS}
        The package can be launched as a ROS node
    \end{alertblock}
    \begin{alertblock}{Integration with robocup repository}
    \begin{itemize}
        \item can run with move\_base
        \item can load environments from the robocup repository
    \end{itemize}
    \end{alertblock}
\end{frame}

%-------------------------------------------------
%   Memory model
%-------------------------------------------------
\begin{frame}{Memory model}
    \begin{alertblock}{MongoDB}
        \begin{itemize}
            \item The maps are stored in a local MongoDB server
            \item Adding map can be done in RVIZ by launching the map setting launch file and the corresponding rviz configuration file in the package.
            \item 
        \end{itemize}
    \end{alertblock}
\end{frame}

%-------------------------------------------------
%   END
%-------------------------------------------------
\end{document}
